\chapter{Data}
\label{cha:Data}

\section{Prompt Dataset}

The prompt analysis requires a dataset of user generated diffusion model prompts. Such a dataset can be collected from discord servers, the services' websites or dedicated prompt and image hostig websites such as 'Lexica Art' \autocite{lexica}. Ideally the dataset should contain prompts from different diffusion services such as 'Stable Diffusion' and 'Midjourney' to allow for a comparison between them. Differences in prompt structure for different services have already been observed by \autocite{ramesh}.
A comparison of prompts for different versions of the services could also be possible given such data.

I will use the 'diffusiondb' dataset that is publicly available on huggingface \autocite{poloclub-diffusiondb}, it contains a set of 14 million prompt-image pairs with about 1.8 million unique prompts. The prompt-image pairs were collected from the 'Stable Diffusion' discord server and include some additional metadata such as the prompt author's hashed username, the date it was created, the image generation seed and more.

The dataset's authors \autocite{poloclub-diffusiondb} acknowledge there might be a bias in the prompt collection, as the prompts were collected from the official 'Stable Diffusion' Discord server. This server might have a disproportionate amount of AI art enthusiasts and might not be representative of novice users.


Given that the data consists solely of 'Stable Diffusion' prompts, the findings might not apply to other diffusion services.

The prompts were collected from the ... to the ..August 2022...

\autocite{poloclub-diffusiondb} already did some analysis....

- word cloud of top 40 terms
- distribution of amount of specifier clauses (getrennt durch Komma, Semikolon und |)
- mention Greg Rutkowski for fantasy prompts


TODO check den Satz mit Ramesh (der Inhalt wurde aus dem diffusiondb Paper übernommen)

...

\section{Artist Dataset}

