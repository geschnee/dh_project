\chapter{Einleitung}
\label{cha:Introduction}
% \lipsum \autocite{DBLP:books/sp/HarderR01}

For a long time the creation of computer art was mostly limited to programmers, engineers and scientists, for they had access to computing resources and programming knowledge. Since then the creation of computer art has become much more accessible with drawing and image editing tools. The recent progress of research in the field of artificial intelligence has led to the development of many image generation tools with different underlying algorithms and purposes. One of these are diffusion models, which are a class of generative models that are able to generate images of high quality conditioned on user input. Services such as 'Stable Diffusion' and 'Midjourney' allow users to generate images and art based on images descriptions, also called the prompt, and other parameters, such as for example a reference image.

This new technology has lead to a series of controversies, in August 2022 an image generated by Jason Allen won the Colorado State Fair's digital art competition. It was later revealed to be computer generated, which sparked a discussion about the merit of its creator and artificial intelligence in art \autocite{colorado}. Furthermore these models were trained on a dataset of images collected from the internet, it is currently unclear if cases of copyright infringement have been committed by the model's creators. In response lawsuits have been filed \autocite{getty}.

Nevertheless these services have become very popular, with many users generating images and sharing them on social media and dedicated discord channels. Many of the services have dedicated discord servers or websites, where the issued prompts and generated images are published. Diffusion model users formed communities on the various discords and messageboards like Twitter and Reddit. 


\subsection{Prompt Design}

Users and researchers discuss the images and the prompts used to generate them, from this exchange and the exploration of these services the users and researchers \autocite{design-guidelines} noticed patterns in the prompts. As an example, the users found that adding certain keywords, such as 'unreal engine' to a prompt results in hyperrealistic, 3D render image. Furthermore the usage of artist names can be used to influence the style and content of images \autocite{investigating}.
This process of building prompts to obtain the desired output is called prompt engineering and has been studied for diffusion models \autocite{investigating} as well as other generative models like ChatGPT \autocite{chatgpt-prompt-patterns}. This process requires trial and error since the models are very complex and it currently cannot be explained how different parts of a prompt influence the output.

Many users in the AI art community started mentioning the Polish artist Greg Rutkowski in their prompts, this even lead to some of these generated images appearing in front of his own work in Google searches. The artist's name was used in some early official tutorials and has become a symbol of AI art. This overall popularity of the artist in the scene, might be the reason why his name is mentioned so often in prompts, compared to mentioning him for his unique style. This could also be described as following a trend or cargo cult programming \autocite{ccp}.


In this project I will investigate the way in which diffusion model users reference artists in their prompts.
I will try to find out if the top mentioned artists are mentioned to replicate artworks in their style or if they are rather mentioned for other reasons.
I will investigate this by trying to disprove the following hypothesis:
Top mentioned artists are mentioned specifically to replicate artworks in their style.




TODO motivate research question
TODO mention related work

TODO potential related work (papers that reference the diffusiondb dataset paper or are referenced in it)
% https://scholar.google.de/scholar?cites=18407007433834814811&as_sdt=2005&sciodt=0,5&hl=de
Willison et al. (2022) analogize writing prompts to wizards learning “magical spells”: users do not understand why some prompts work, but they will add these prompts to their “spell book.” For example, to generate highly-detailed images, it has become a common practice to add special keywords such as “trending on artstation” and “unreal engine” in the prompt.
podcast: https://changelog.com/podcast/506



TODO cite the diffusiondb paper, they already did some anlysis

TODO cite https://arxiv.org/abs/2204.13988 text-to-image prompt analyse and classification

TODO why is work relevant? 
--> viele Künstler sind nicht damit einverstanden, dass ihr Kunstwerke für das Training genutzt werden.
--> die Beantwortung der Frage kann für die weiter Argumentation wichtig sein...  

TODO why is Greg mentioned this often:
-> https://www.technologyreview.com/2022/09/16/1059598/this-artist-is-dominating-ai-generated-art-and-hes-not-happy-about-it/
--> was used as an example in early tutorial % https://colab.research.google.com/github/alembics/disco-diffusion/blob/main/Disco_Diffusion.ipynb
-----> popularity boost
-> %https://www.reddit.com/r/StableDiffusion/comments/wze2s3/who_is_greg_rutkowski_and_why_does_every_prompt/
--> I think there's a bit of **cargo culting** in that people include his name because they see it in prompts that they like, regardless of how much of an impact his name has on the end result.
-> https://news.artnet.com/art-world/a-i-should-exclude-living-artists-from-its-database-says-one-painter-whose-works-were-used-to-fuel-image-generators-2178352

TODO:
Each project needs a research question that can be
investigated by means of some data and an analytical
method. The research question is typically motivated by
some related work (quote at least 4-5 relevant papers).
Also show what others have done in this research area
and show why your research question is relevant. Possibly
also highlight the research gap (what’s new?) that you plan
to fill.


TODO cite SD and Midjourney and the original first paper of diffusion models

% andere interessante Links:
% https://www.reddit.com/r/StableDiffusion/comments/zo95xi/greg_rutkowski_just_posted_the_no_ai_image_on_his/

% TODO " vs ' im fertigen Dokument prüfen
