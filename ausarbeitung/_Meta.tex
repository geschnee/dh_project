% Informationen ------------------------------------------------------------
% 	Definition von globalen Parametern, die im gesamten Dokument verwendet
% 	werden können (z.B auf dem Deckblatt etc.).
% --------------------------------------------------------------------------
\newcommand{\titel}{Investigating artist and style mentions in Diffusion Model prompts}
\newcommand{\art}{Projektarbeit Digital Humanities}
\newcommand{\ort}{Leipzig}
\newcommand{\hochschule}{Universität Leipzig}
\newcommand{\fachgebiet}{Digital Humanities}
\newcommand{\fakultaet}{Fakultät für Mathematik und Informatik}
\newcommand{\institut}{Institut für Informatik}
\newcommand{\autor}{Georg Schneeberger}
\newcommand{\matrikelnr}{3707914}
\newcommand{\erstgutachter}{Dr. Andreas Niekler}
\newcommand{\jahr}{2023}
\newcommand{\eingereicht}{15.03.2023}

% Eigene Befehle
\newcommand{\todo}[1]{\textbf{\textsc{\textcolor{red}{(TODO: #1)}}}}

% Autorennamen in small caps
\newcommand{\AutorZ}[1]{\textsc{#1}}
\newcommand{\Autor}[1]{\AutorZ{\citeauthor{#1}}}

% Befehle zur semantischen Auszeichnung von Text
\newcommand{\NeuerBegriff}[1]{\textbf{#1}}
\newcommand{\Fachbegriff}[1]{\textit{#1}}
\newcommand{\Prozess}[1]{\textit{#1}}
\newcommand{\Webservice}[1]{\textit{#1}}
\newcommand{\Eingabe}[1]{\texttt{#1}}
\newcommand{\Code}[1]{\texttt{#1}}
\newcommand{\Datei}[1]{\texttt{#1}}
\newcommand{\Datentyp}[1]{\textsf{#1}}
\newcommand{\XMLElement}[1]{\textsf{#1}}

% Abkürzungen
\newcommand{\vgl}{Vgl.\ }
\newcommand{\ua}{\mbox{u.\,a.\ }}
\newcommand{\zB}{\mbox{z.\,B.\ }}
\newcommand{\bs}{$\backslash$}

% Einfache Anführungszeichen in texttt
\newcommand{\sq}{\textquotesingle}

